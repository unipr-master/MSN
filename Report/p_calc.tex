\subsection{Calcolo di $P_{1}$}

La quantità $P_1$ viene calcolata come rapporto tra la taglia del cluster più grande ($s_{max}$) e il numero totale di siti presenti nel reticolo ($L^2$):
\vspace{4px}
\begin{equation}
	P_1 = \frac{s_{max}}{L^2}
\end{equation}

\vspace{4px}
\noindent
Dal punto di vista probabilistico, $P_1$ rappresenta la frazione dell'intero reticolo occupata dal cluster dominante. Quando $p_{col}$ è molto basso, i cluster tendono ad essere piccoli e isolati, quindi $P_1$ assume valori trascurabili. Ma man mano che $p_{col}$ si avvicina alla soglia critica, $P_1$ cresce rapidamente: questo riflette la formazione di cluster massicci, capaci di connettere porzioni opposte del reticolo. Di seguito sono riportati i valori ottenuti in fase di simulazione

\vspace{15px}
\noindent
\begin{tabular}{|c|*{11}{c|}}
	\hline
	\textbf{} & \textbf{0.55} & \textbf{0.56} & \textbf{0.57} & \textbf{0.58} & \textbf{0.59} & \textbf{0.60} & \textbf{0.61} & \textbf{0.62} & \textbf{0.63} & \textbf{0.64} & \textbf{0.65} \\
	\hline
	\textbf{100}  & 0.07 & 0.10 & 0.13 & 0.16 & 0.26 & 0.33 & 0.41 & 0.48 & 0.53 & 0.57 & 0.60 \\
	\hline
	\textbf{300}  & 0.02 & 0.03 & 0.05 & 0.08 & 0.19 & 0.37 & 0.46 & 0.54 & 0.57 & 0.59 & 0.61 \\
	\hline
	\textbf{1000} & 0.00 & 0.00 & 0.01 & 0.03 & 0.12 & 0.42 & 0.51 & 0.55 & 0.58 & 0.60 & 0.62 \\
	\hline
\end{tabular}

\vspace{15px}
\noindent
e i relativi errori

\vspace{15px}
\noindent
\begin{tabular}{|c|*{11}{c|}}
	\hline
	\textbf{} & \textbf{0.55} & \textbf{0.56} & \textbf{0.57} & \textbf{0.58} & \textbf{0.59} & \textbf{0.60} & \textbf{0.61} & \textbf{0.62} & \textbf{0.63} & \textbf{0.64} & \textbf{0.65} \\
	\hline
	\textbf{100}  & 0.00 & 0.01 & 0.01 & 0.01 & 0.01 & 0.01 & 0.01 & 0.01 & 0.01 & 0.00 & 0.00 \\
	\hline
	\textbf{300}  & 0.00 & 0.00 & 0.00 & 0.01 & 0.01 & 0.01 & 0.01 & 0.00 & 0.00 & 0.00 & 0.00 \\
	\hline
	\textbf{1000} & 0.00 & 0.00 & 0.00 & 0.00 & 0.01 & 0.01 & 0.00 & 0.00 & 0.00 & 0.00 & 0.00 \\
	\hline
\end{tabular}

\subsection{Calcolo di $P_{2}$}

La seconda quantità, $P_2$, mantiene lo stesso numeratore ($s_{max}$) ma usa come denominatore il valore atteso dei siti occupati ($p_{col} \cdot L^2$):

\vspace{4px}
\begin{equation}
P_2 = \frac{s_{max}}{p_{col} \cdot L^2}
\end{equation}

\vspace{4px}
\noindent
Questa formulazione permette di interpretare $P_2$ come la frazione dei siti che, in media, ci si aspetta siano colorati e che appartengono al cluster più grande. In altre parole, si misura quanto il cluster principale pesa rispetto al totale dei siti che dovrebbero essere colorati. Valori alti di $P_2$ indicano che il cluster dominante raccoglie una parte significativa della massa colorata. Anche $P_2$, come le altre grandezze, mostra una variazione brusca in prossimità della soglia di percolazione, rendendolo un buon indicatore del passaggio alla fase percolante. Di seguito sono riportati i valori ottenuti in fase di simulazione.

\vspace{15px}
\noindent
\begin{tabular}{|c|*{11}{c|}}
	\hline
	\textbf{} & \textbf{0.55} & \textbf{0.56} & \textbf{0.57} & \textbf{0.58} & \textbf{0.59} & \textbf{0.60} & \textbf{0.61} & \textbf{0.62} & \textbf{0.63} & \textbf{0.64} & \textbf{0.65} \\
	\hline
	\textbf{100}  & 0.12 & 0.18 & 0.23 & 0.28 & 0.44 & 0.55 & 0.68 & 0.77 & 0.85 & 0.89 & 0.92 \\
	\hline
	\textbf{300}  & 0.03 & 0.05 & 0.08 & 0.14 & 0.33 & 0.61 & 0.75 & 0.86 & 0.90 & 0.93 & 0.94 \\
	\hline
	\textbf{1000} & 0.01 & 0.01 & 0.02 & 0.05 & 0.20 & 0.71 & 0.83 & 0.88 & 0.91 & 0.93 & 0.95 \\
	\hline
\end{tabular}

\vspace{15px}
\noindent
e i relativi errori

\vspace{15px}
\noindent
\begin{tabular}{|c|*{11}{c|}}
	\hline
	\textbf{} & \textbf{0.55} & \textbf{0.56} & \textbf{0.57} & \textbf{0.58} & \textbf{0.59} & \textbf{0.60} & \textbf{0.61} & \textbf{0.62} & \textbf{0.63} & \textbf{0.64} & \textbf{0.65} \\
	\hline
	\textbf{100}  & 0.01 & 0.01 & 0.01 & 0.02 & 0.02 & 0.02 & 0.02 & 0.02 & 0.01 & 0.01 & 0.00 \\
	\hline
	\textbf{300}  & 0.00 & 0.00 & 0.01 & 0.01 & 0.02 & 0.02 & 0.01 & 0.00 & 0.00 & 0.00 & 0.00 \\
	\hline
	\textbf{1000} & 0.00 & 0.00 & 0.00 & 0.00 & 0.01 & 0.01 & 0.00 & 0.00 & 0.00 & 0.00 & 0.00 \\
	\hline
\end{tabular}

\subsection{Calcolo di $P_{3}$}

Nel calcolo di $P_3$, il numeratore è sempre $s_{max}$, ma il denominatore cambia nuovamente: si utilizza il numero effettivo di siti colorati nel reticolo, cioè la somma delle taglie di tutti i cluster presenti:

\vspace{4px}
\begin{equation}
P_3 = \frac{s_{max}}{\sum_{s} s \cdot n_s}
\end{equation}

\vspace{4px}
\noindent
Questa quantità riflette la frazione di siti occupati che appartengono al solo cluster dominante. È quindi una misura “interna” al sistema: tra tutti i siti effettivamente colorati, quanti sono coinvolti nel cluster principale? Per valori bassi di $p_{col}$, il sistema è disperso e $P_3$ è piccolo. Attorno alla soglia, il valore cresce bruscamente, segnalando che il cluster dominante comincia ad assorbire gran parte dei siti colorati. Sopra la soglia, $P_3$ tende rapidamente a 1. Di seguito sono riportati i valori ottenuti in fase di simulazione

\vspace{15px}
\noindent
\begin{tabular}{|c|*{11}{c|}}
	\hline
	\textbf{} & \textbf{0.55} & \textbf{0.56} & \textbf{0.57} & \textbf{0.58} & \textbf{0.59} & \textbf{0.60} & \textbf{0.61} & \textbf{0.62} & \textbf{0.63} & \textbf{0.64} & \textbf{0.65} \\
	\hline
	\textbf{100}  & 0.13 & 0.17 & 0.21 & 0.29 & 0.41 & 0.53 & 0.68 & 0.79 & 0.85 & 0.89 & 0.92 \\
	\hline
	\textbf{300}  & 0.03 & 0.05 & 0.08 & 0.15 & 0.31 & 0.61 & 0.79 & 0.86 & 0.90 & 0.92 & 0.94 \\
	\hline
	\textbf{1000} & 0.01 & 0.01 & 0.02 & 0.05 & 0.22 & 0.70 & 0.83 & 0.88 & 0.91 & 0.93 & 0.95 \\
	\hline
\end{tabular}

\vspace{15px}
\noindent
e i relativi errori

\vspace{15px}
\noindent
\begin{tabular}{|c|*{11}{c|}}
	\hline
	\textbf{} & \textbf{0.55} & \textbf{0.56} & \textbf{0.57} & \textbf{0.58} & \textbf{0.59} & \textbf{0.60} & \textbf{0.61} & \textbf{0.62} & \textbf{0.63} & \textbf{0.64} & \textbf{0.65} \\
	\hline
	\textbf{100}  & 0.01 & 0.01 & 0.01 & 0.02 & 0.02 & 0.02 & 0.02 & 0.02 & 0.01 & 0.00 & 0.00 \\
	\hline
	\textbf{300}  & 0.00 & 0.00 & 0.01 & 0.01 & 0.02 & 0.02 & 0.01 & 0.00 & 0.00 & 0.00 & 0.00 \\
	\hline
	\textbf{1000} & 0.00 & 0.00 & 0.00 & 0.00 & 0.01 & 0.01 & 0.00 & 0.00 & 0.00 & 0.00 & 0.00 \\
	\hline
\end{tabular}

\vspace{15px}
\noindent
Questa crescita è indicativa del passaggio da una fase dominata da piccoli cluster a una dominata da un solo cluster, molto esteso.

\subsection{Calcolo di $RACS$}

La misura $RACS$ (Reduced Average Cluster Size) viene calcolata secondo la formula:

\vspace{4px}
\begin{equation}
RACS= \frac{\sum\limits_{s \neq s_{max}} s \cdot (s \cdot n_s)}{\sum\limits_{s'} s' \cdot n_{s'}}
\end{equation}

\vspace{4px}
\noindent
Al numeratore si ha una media pesata delle taglie dei cluster, \textbf{escludendo} il cluster più grande, mentre al denominatore si somma la massa totale dei siti colorati (incluso il cluster massimo). In pratica, $RACS$ misura la taglia media dei cluster piccoli, rapportata alla "massa" complessiva del sistema.
\\\\
\noindent
Il fatto che il numeratore escluda $s_{max}$ è importante: attorno alla soglia, il cluster dominante tende ad assorbire una quantità crescente di siti, distorcendo il comportamento degli altri cluster. Rimuovendo il suo contributo, $RACS$ permette di isolare il comportamento del resto della rete di cluster.
\\\\
\noindent
Dal punto di vista interpretativo, si osserva che $RACS$ cresce con $p_{col}$, raggiunge un massimo nei pressi della soglia di percolazione e poi decresce, perché il cluster più grande inizia a dominare completamente il sistema. Di seguito sono riportati i valori ottenuti in fase di simulazione

\vspace{15px}
\noindent
\begin{tabular}{|c|*{11}{c|}}
	\hline
	\textbf{} & \textbf{0.55} & \textbf{0.56} & \textbf{0.57} & \textbf{0.58} & \textbf{0.59} & \textbf{0.60} & \textbf{0.61} & \textbf{0.62} & \textbf{0.63} & \textbf{0.64} & \textbf{0.65} \\
	\hline
	\textbf{100}  & 280 & 429 & 435 & 592 & 624 & 648 & 410 & 286 & 164 & 127 & 104 \\
	\hline
	\textbf{300}  & 669 & 1021 & 1712 & 3354 & 5157 & 3909 & 1572 & 426 & 306 & 136 & 109 \\
	\hline
	\textbf{1000} & 1053 & 1852 & 3799 & 10714 & 47874 & 17281 & 1961 & 649 & 322 & 202 & 154 \\
	\hline
\end{tabular}

\vspace{15px}
\noindent
e i relativi errori

\vspace{15px}
\noindent
\begin{tabular}{|c|*{11}{c|}}
	\hline
	\textbf{} & \textbf{0.55} & \textbf{0.56} & \textbf{0.57} & \textbf{0.58} & \textbf{0.59} & \textbf{0.60} & \textbf{0.61} & \textbf{0.62} & \textbf{0.63} & \textbf{0.64} & \textbf{0.65} \\
	\hline
	\textbf{100}  & 14 & 23 & 27 & 39 & 53 & 57 & 60 & 63 & 24 & 18 & 25 \\
	\hline
	\textbf{300}  & 18 & 43 & 66 & 201 & 461 & 531 & 401 & 47 & 60 & 14 & 11 \\
	\hline
	\textbf{1000} & 14 & 40 & 150 & 481 & 3429 & 3311 & 258 & 64 & 29 & 18 & 14 \\
	\hline
\end{tabular}

\vspace{15px}
\noindent
In un reticolo infinito, dove i concetti di percolazione da un bordo all’altro non sono più applicabili, $RACS$ aiuta a descrivere la transizione in termini statistici: indica quando la distribuzione dei cluster cambia bruscamente e un unico cluster diventa macroscopico.